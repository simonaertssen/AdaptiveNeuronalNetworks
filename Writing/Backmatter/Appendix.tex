% !TEX root = ../main.tex

\newpage
\appendix
\section{Appendix} \label{sec:Appendix}

\subsection{Transformation to the QIF model} \label{app:TransformationToQIF}
We prove that the transformation \eqref{eq:QIFtransformation} holds from the \QIF model \eqref{eq:QIFmodel} to the Theta model \eqref{eq:thetaneuron}.
\begin{align*}
V &\equiv \tan \left( \frac{\theta}{2} \right) \quad \longrightarrow \quad
\frac{\mathop{d V}}{\mathop{d t}} = \frac{1}{2 \cos ^{2}\left(\frac{\theta}{2}\right)} \frac{d \theta}{ \mathop{d t}}
\end{align*}
Insert into $\frac{\mathop{d V}}{\mathop{d t}}= V^2 + I$:
\begin{align*}
\frac{\mathop{d \theta}}{\mathop{d t}} &= 2\left(\cos ^{2}\left(\frac{\theta}{2}\right) \cdot \tan ^{2}\left(\frac{\theta}{2}\right)+\cos ^{2}\left(\frac{\theta}{2}\right) \cdot I \right) = 2\left(\sin ^{2}\left(\frac{\theta}{2}\right)+\cos ^{2}\left(\frac{\theta}{2}\right) \cdot I \right)
\end{align*}
Using $\cos ^{2}\left(\frac{\theta}{2}\right) = \frac{1+\cos \left(\frac{\theta}{2}\right)}{2}$ and $\sin ^{2}\left(\frac{\theta}{2}\right)=\frac{1-\cos \left(\frac{\theta}{2}\right)}{2}$:
\begin{align*}
\dot{\theta} &=2\left(\frac{1-\cos \theta}{2}+\left(\frac{1+\cos \theta}{2}\right) \cdot I \right) =(1-\cos \theta)+(1+\cos \theta) \cdot I
\end{align*}
This proves that the transformation \eqref{eq:QIFtransformation} is correct.

 
\subsection{Solutions to the QIF model} \label{app:ThetaModelSolutions}
Depending on the value of $I$, we can distinguish multiple solutions  \cite{Perez2020}. In all cases we can integrate through the separation of variables. Solutions are bound to start at $V(t_0)$, right after a spike has occured at $t=t_0$. 

\subsubsection*{Solving for \texorpdfstring{$I < 0$}{TEXT}}
%Change variables: $I = -\tilde{I}^2$. Separate variables as $\frac{\mathop{dv}}{v^2 - \tilde{I}^2} = \mathop{dt}$ and integrate from $t = 0$:
\begin{align*}
\int_{V(t_0)}^{V(t)} \frac{\mathop{dv}}{v^2 - \tilde{I}^2} &= \int_{V(t_0)}^{V(t)} \frac{\mathop{dv}}{(v+\tilde{I})(v-\tilde{I})} 
= \frac{1}{2 \tilde{I}} \int_{V(t_0)}^{V(t)} \frac{\mathop{dv}}{v-\tilde{I}}-\frac{1}{2 \tilde{I}} \int_{V(t_0)}^{V(t)} \frac{\mathop{dv}}{v+\tilde{I}} \\
%&= \frac{1}{2 \tilde{I}} \log (v-\tilde{I}) -\frac{1}{2 \tilde{I}} \log (v+\tilde{I})\Big \rvert_{V(t_0)}^{V(t)} 
&= \frac{1}{2 \tilde{I}} \log \left(1-\frac{2 \tilde{I}}{v+\tilde{I}}\right) \Big \rvert_{V(t_0)}^{V(t)} 
= \int_{t_0}^t \mathop{d\tau} = t - t_0 \\
V(t) &= \lim_{V(t_0) \rightarrow -\infty} \frac{2 \sqrt{-I}}{1 - \left(1-\frac{2 \sqrt{-I}}{V(t_0)+\sqrt{-I}}\right)\cdot e^{2 (t - t_0)\sqrt{-I}}}-\sqrt{-I}\\
&= \frac{2 \sqrt{-I}}{1 - e^{2 (t - t_0) \sqrt{-I}}}-\sqrt{-I}
\end{align*}

\subsubsection{Solving for \texorpdfstring{$I = 0$}{TEXT}}
\begin{align*}
\int_{V(t_0)}^{V(t)} \frac{\mathop{dv}}{v^2} &= \frac{1}{v}\Big\rvert_{V(t_0)}^{V(t)} = - \frac{1}{V(t)} + \frac{1}{V(t_0)} = \int_{t_0}^t \mathop{d\tau} = t -t_0 \\
V(t) &= \lim_{V(t_0) \rightarrow - \infty} \frac{V(t_0)}{1-V(t_0)(t - t_0)} \underset{\frac{\infty}{\infty}}{\overset{\mathrm{H}}{=}} \frac{-1}{t - t_0}
\end{align*}

\subsubsection{Solving for \texorpdfstring{$I > 0$}{TEXT}}
%Separate the variables as $\frac{\mathop{dv}}{v^2 + I} = \mathop{dt}$ and integrate from $t=0$:
\begin{align*}
\int_{V(t_0)}^{V(t)} \frac{\mathop{dv}}{v^2 + I} &= \int_{V(t_0)}^{V(t)} \frac{I}{\left(\frac{v}{\sqrt{I}}\right)^2 + 1} \mathop{dv} 
\overset{x = \frac{v}{\sqrt{I}}}{\underset{\mathop{dx} = \frac{dv}{\sqrt{I}}}{=}} 
%\quad \longrightarrow \quad x = \frac{v}{\sqrt{I}} \quad \mathop{dx} = \frac{\mathop{dv}}{\sqrt{I}} \\
\int_{\frac{V(t_0)}{\sqrt{I}}}^{\frac{V(t)}{\sqrt{I}}} \frac{I}{x^2 + 1} \mathop{dx}= \frac{1}{\sqrt{I}} \arctan(x) \Big \rvert_{\frac{V(t_0)}{\sqrt{I}}}^{\frac{V(t)}{\sqrt{I}}} \\
&= \frac{1}{\sqrt{I}} \left( \arctan \left( \frac{V(t)}{\sqrt{I}} \right) - \arctan \left( \frac{V(t_0)}{\sqrt{I}} \right) \right) = 
\int_{t_0}^t \mathop{d\tau} = t - t_0 \\
V(t) &= \lim_{V(t_0) \rightarrow -\infty} \sqrt{I} \cdot \tan \left( (t - t_0) \sqrt{I} + \arctan \left( \frac{V(t_0)}{\sqrt{I}} \right) \right) = \sqrt{I} \cdot \tan \left( (t - t_0) \sqrt{I} - \frac{\pi}{2} \right) \\
&=  \sqrt{I} \cdot \cot \left( (t - t_0) \sqrt{I} \right) 
\end{align*}


\subsection{Frequency response of the neuron models} \label{app:ThetaModelFrequencyResponse}
The integral is solved like before, but now with the conditions of the spike:
\begin{align*}
T &= \lim_{a \rightarrow \infty} \int_{-a}^{a} \frac{I}{\left(\frac{v}{\sqrt{I}}\right)^2 + 1} \mathop{dv} 
\overset{x = \frac{v}{\sqrt{I}}}{\underset{\mathop{dx} = \frac{dv}{\sqrt{I}}}{=}} 
\lim_{a \rightarrow \infty} \int_{\frac{-a}{\sqrt{I}}}^{\frac{a}{\sqrt{I}}} \frac{I}{x^2 + 1} \mathop{dx}
= \lim_{a \rightarrow \infty} \frac{1}{\sqrt{I}} \arctan(x) \Big \rvert_{\frac{-a}{\sqrt{I}}}^{\frac{a}{\sqrt{I}}} \\
&= \frac{1}{\sqrt{I}} \left( \frac{\pi}{2} - \left( - \frac{\pi}{2} \right) \right)
= \frac{\pi}{\sqrt{I}}
\end{align*}
So the frequency of oscillation is proportional to $\sqrt{I}$. 


\subsection{Newton-Raphson root iteration} \label{app:NewtonRaphson}
We define the equilibria $\boldsymbol{x^\ast} \in \R^n$ of a multivariate function $\boldsymbol{f}(\boldsymbol{x}) : \R^n \rightarrow \R^n$ with $\boldsymbol{f}(\boldsymbol{x}) = \boldsymbol{0}$. Expanding $\boldsymbol{f}$ as a Taylor series, we obtain:
\begin{align*}
f_i(\boldsymbol{x} + \delta \boldsymbol{x}) =f_{i}(\boldsymbol{x}) + \sum_{j=1}^{n} \frac{\partial f_{i}(\boldsymbol{x})}{\partial x_{j}} \delta x_{j}+O\left(\delta \boldsymbol{x}^{2}\right) \approx f_{i}(\boldsymbol{x})+\sum_{j=1}^{n} \frac{\partial f_{i}(\boldsymbol{x})}{\partial x_{j}} \delta x_{j}, \qquad (i=1, \cdots, n)
\end{align*}
We can also write this in vector notation, by setting $\boldsymbol{J}(\boldsymbol{x}) = \nabla \boldsymbol{f}(\boldsymbol{x}) = \frac{d}{d\boldsymbol{x}} \boldsymbol{f}(\boldsymbol{x}) \in \R^{n \times n}$ 
\begin{align*}
\boldsymbol{f}(\boldsymbol{x}+\delta \boldsymbol{x}) &\approx\left[\begin{array}{c}f_{1}(\boldsymbol{x}) \\ \vdots \\ f_{N}(\boldsymbol{x})\end{array}\right] 
+ \left[\begin{array}{ccc}\frac{\partial f_{1}}{\partial x_{1}} & \cdots & \frac{\partial f_{1}}{\partial x_{N}} \\ \vdots & \ddots & \vdots \\ \frac{\partial f_{N}}{\partial x_{1}} & \cdots & \frac{\partial f_{N}}{\partial x_{N}}\end{array}\right]
\left[\begin{array}{c}\delta x_{1} \\ \vdots \\ \delta x_{N}\end{array}\right] 
=\boldsymbol{f}(\boldsymbol{x})+\boldsymbol{J}(\boldsymbol{x}) \delta \boldsymbol{x} 
\end{align*}
By assuming $\boldsymbol{f}(\boldsymbol{x}+\delta \boldsymbol{x}) = 0$ we can find that $\delta \boldsymbol{x} = -\boldsymbol{J}^{-1}( \boldsymbol{x}) \boldsymbol{f}(\boldsymbol{x})$ so that $\boldsymbol{x} + \delta \boldsymbol{x} =  \boldsymbol{x} - \boldsymbol{J}^{-1} (\boldsymbol{x}) \boldsymbol{f}(\boldsymbol{x})$. This expression converges to $\boldsymbol{x^\ast}$. When the equations are nonlinear, the equations converge to the real root as $\boldsymbol{x}_k =  \boldsymbol{x}_k - \boldsymbol{J}^{-1} ( \boldsymbol{x}_k)\boldsymbol{f}(\boldsymbol{x}_k)$.


\subsection{Jacobian of the Ott-Antonsen manifold}
Starting from \eqref{eq:OttAntonsenSystemFull}, we separate the real and imaginary parts of $z(\k, t)$ in $x_{\k} = x(\k, t)$ and $y_{\k} = y(\k, t)$:
\begin{align*}
\frac{\partial z_{\k}}{\partial t} &= -\frac{\ic}{2} \cdot \left( x_{\k}^2 + \ic 2x_{\k}y_{\k} - y_{\k}^2 - 2x_{\k} - \ic 2y_{\k} +1\right) + \frac{1}{2} \cdot \left( x_{\k}^2 + \ic 2x_{\k}y_{\k} - y_{\k}^2 + 2x_{\k} + \ic 2y_{\k} +1 \right) \cdot I_{\k}\\
I_{\k} &= -\sigma_{\k} + \ic \eta_{0, \k} + \ic H_{2_{\k}} \\
H_{2_{\k}} &= \frac{\kappa}{\kmean} \sum_{\kacc \in \K} P\left(\kacc\right) \: a\left(\kacc \rightarrow \k\right) \cdot \left( 1 + \frac{x_{\k}^2}{3} - \frac{4}{3} x_{\k} \right)
\end{align*}
Taking the dynamics per real and imaginary value yields:
\begin{align*}
\frac{\partial x_{\k}}{\partial t}
&=f_{\k} \left(x_{\k}, y_{\k} \right) \\ 
&=(x_{\k}-1) y_{\k}-\frac{(x_{\k}+1)^{2}-y_{\k}^{2}}{2} \sigma_{\k} + (x_{\k}+1) y_{\k} \left[\eta_{0}+ H_{2_{\k}}\right] \\ 
\frac{\partial x_{\k}}{\partial t}
&= g_{\k} \left(y_{\k}, y_{\k} \right) \\ 
&=-\frac{(x_{\k}-1)^{2}-y_{\k}^{2}}{2}-(x_{\k}+1) y_{\k} \sigma_{\k} + \frac{(x_{\k}+1)^{2}-y_{\k}^{2}}{2} \cdot \left[\eta_{0} + H_{2_{\k}} \right] 
\end{align*}
And the Jacobian is found from the partial derivatives of $f$ and $g$:
\begin{align*}
\frac{\partial f_{\k}}{\partial x_{\k}} &= y_{\k}-(x_{\k}+1) \sigma_{\k}-y_{\k} \left[\eta_{0} + \kappa \cdot H_{2_{\k}} \right] + (x_{\k}+1) y_{\k} \frac{\partial H_{2_{\k}}}{\partial x_{\k}}\\
\frac{\partial f_{\k}}{\partial y_{\k}} &= (x_{\k}-1)+y_{\k} \sigma_{\k} + (x_{\k}+1) \left[\eta_{0} + H_{2_{\k}} \right] \\
\frac{\partial g_{\k}}{\partial x_{\k}} &= -(x_{\k} - 1) - y_{\k}  \sigma_{\k} + (x_{\k} + 1) \left[\eta_{0} + H_{2_{\k}} \right] + \left( \frac{(x_{\k} + 1)^2 - y_{\k}^2}{2} \right) \frac{\partial H_{2_{\k}}}{\partial x_{\k}} \\
\frac{\partial g_{\k}}{\partial y_{\k}} &= y_{\k} - (x_{\k} + 1) \sigma_{\k} - y_{\k} \left[\eta_{0} + H_{2_{\k}} \right] \\
\frac{\partial H_{2_{\k}}}{\partial x_{\k}} &= \frac{\kappa}{\langle k\rangle} P\left(\k\right) \: a\left(\k \rightarrow \k\right) (x_{\k}-2) \frac{2}{3} \\
\frac{\partial H_{2_{\k}}}{\partial y_{\k}} &= 0 
\end{align*}

And the off-diagonal elements, the nodes represented by degree $\kacc$:
\begin{align*}
\frac{\partial f_{\k}}{\partial x_{\kacc}} &= (x_{\k}+1) y_{\k} \frac{\partial H_{2_{\k}}}{\partial x_{\kacc}}\\
\frac{\partial f_{\k}}{\partial y_{\kacc}} &= 0 \\
\frac{\partial g_{\k}}{\partial x_{\kacc}} &= \left( \frac{(x_{\k} + 1)^2 - y_{\k}^2}{2} \right) \frac{\partial H_{2_{\k}}}{\partial x_{\kacc}} \\
\frac{\partial g_{\k}}{\partial y_{\kacc}} &= 0\\
\frac{\partial H_{2_{\k}}}{\partial x_{\kacc}} &= \frac{\kappa}{\langle k\rangle} P\left(\kacc\right) \: a\left(\kacc \rightarrow \k\right)(x_{\kacc}-2) \frac{2}{3} \\
\frac{\partial H_{2_{\k}}}{\partial y_{\kacc}} &= 0
\end{align*}


\label{LastPage}~





