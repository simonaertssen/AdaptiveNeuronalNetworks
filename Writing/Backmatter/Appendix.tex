% !TEX root = ../main.tex

\newpage
\appendix
\section{Appendix} \label{sec:Appendix}

\subsection{Transformation to the QIF model} \label{app:TransformationToQIF}
We prove that the transformation \eqref{eq:QIFtransformation} holds from the \QIF model \eqref{eq:QIFmodel} to the Theta model \eqref{eq:thetaneuron}.
\begin{align*}
V &\equiv \tan \left( \frac{\theta}{2} \right) \quad \longrightarrow \quad
\frac{\mathop{d V}}{\mathop{d t}} = \frac{1}{2 \cos ^{2}\left(\frac{\theta}{2}\right)} \frac{d \theta}{ \mathop{d t}}
\end{align*}
Insert into $\frac{\mathop{d V}}{\mathop{d t}}= V^2 + I$:
\begin{align*}
\frac{\mathop{d \theta}}{\mathop{d t}} &= 2\left(\cos ^{2}\left(\frac{\theta}{2}\right) \cdot \tan ^{2}\left(\frac{\theta}{2}\right)+\cos ^{2}\left(\frac{\theta}{2}\right) \cdot I \right) = 2\left(\sin ^{2}\left(\frac{\theta}{2}\right)+\cos ^{2}\left(\frac{\theta}{2}\right) \cdot I \right)
\end{align*}
Using $\cos ^{2}\left(\frac{\theta}{2}\right) = \frac{1+\cos \left(\frac{\theta}{2}\right)}{2}$ and $\sin ^{2}\left(\frac{\theta}{2}\right)=\frac{1-\cos \left(\frac{\theta}{2}\right)}{2}$:
\begin{align*}
\dot{\theta} &=2\left(\frac{1-\cos \theta}{2}+\left(\frac{1+\cos \theta}{2}\right) \cdot I \right) =(1-\cos \theta)+(1+\cos \theta) \cdot I
\end{align*}
This proves that the transformation \eqref{eq:QIFtransformation} is correct.

 
\subsection{Solutions to the QIF model} \label{app:ThetaModelSolutions}
Depending on the value of $I$, we can distinguish multiple solutions  \cite{Perez2020}. In all cases we can integrate through the separation of variables. Solutions are bound to start at $V(0)$, right after a spike has occured at $t=0$. 

\subsubsection{Solving for \texorpdfstring{$I < 0$}{TEXT}}
%Change variables: $I = -\tilde{I}^2$. Separate variables as $\frac{\mathop{dv}}{v^2 - \tilde{I}^2} = \mathop{dt}$ and integrate from $t = 0$:
\begin{align*}
\int_{V(0)}^{V(t)} \frac{\mathop{dv}}{v^2 - \tilde{I}^2} &= \int_{V(0)}^{V(t)} \frac{\mathop{dv}}{(v+\tilde{I})(v-\tilde{I})} 
= \frac{1}{2 \tilde{I}} \int_{V(0)}^{V(t)} \frac{\mathop{dv}}{v-\tilde{I}}-\frac{1}{2 \tilde{I}} \int_{V(0)}^{V(t)} \frac{\mathop{dv}}{v+\tilde{I}} \\
%&= \frac{1}{2 \tilde{I}} \log (v-\tilde{I}) -\frac{1}{2 \tilde{I}} \log (v+\tilde{I})\Big \rvert_{V(0)}^{V(t)} 
&= \frac{1}{2 \tilde{I}} \log \left(1-\frac{2 \tilde{I}}{v+\tilde{I}}\right) \Big \rvert_{V(0)}^{V(t)} 
= \int_0^t \mathop{d\tau} = t \\
V(t) &= \lim_{V(0) \rightarrow -\infty} \frac{2 \sqrt{-I}}{1 - \left(1-\frac{2 \sqrt{-I}}{V(0)+\sqrt{-I}}\right)\cdot e^{2 t \sqrt{-I}}}-\sqrt{-I}\\
&= \frac{2 \sqrt{-I}}{1 - e^{2 t \sqrt{-I}}}-\sqrt{-I}
\end{align*}

\subsubsection{Solving for \texorpdfstring{$I = 0$}{TEXT}}
\begin{align*}
\int_{V(0)}^{V(t)} \frac{\mathop{dv}}{v^2} &= \frac{1}{v}\Big\rvert_{V(0)}^{V(t)} = - \frac{1}{V(t)} + \frac{1}{V(0)} = \int_0^t \mathop{d\tau} = t \\
V(t) &= \lim_{V(0) \rightarrow - \infty} \frac{V(0)}{1-V(0)t} \underset{\frac{\infty}{\infty}}{\overset{\mathrm{H}}{=}} \frac{-1}{t}
\end{align*}

\subsubsection{Solving for \texorpdfstring{$I > 0$}{TEXT}}
%Separate the variables as $\frac{\mathop{dv}}{v^2 + I} = \mathop{dt}$ and integrate from $t=0$:
\begin{align*}
\int_{V(0)}^{V(t)} \frac{\mathop{dv}}{v^2 + I} &= \int_{V(0)}^{V(t)} \frac{I}{\left(\frac{v}{\sqrt{I}}\right)^2 + 1} \mathop{dv} 
\overset{x = \frac{v}{\sqrt{I}}}{\underset{\mathop{dx} = \frac{dv}{\sqrt{I}}}{=}} 
%\quad \longrightarrow \quad x = \frac{v}{\sqrt{I}} \quad \mathop{dx} = \frac{\mathop{dv}}{\sqrt{I}} \\
\int_{\frac{V(0)}{\sqrt{I}}}^{\frac{V(t)}{\sqrt{I}}} \frac{I}{x^2 + 1} \mathop{dx}= \frac{1}{\sqrt{I}} \arctan(x) \Big \rvert_{\frac{V(0)}{\sqrt{I}}}^{\frac{V(t)}{\sqrt{I}}} \\
&= \frac{1}{\sqrt{I}} \left( \arctan \left( \frac{V(t)}{\sqrt{I}} \right) - \arctan \left( \frac{V(0)}{\sqrt{I}} \right) \right) = 
\int_0^t \mathop{d\tau} &= t \\
V(t) &= \lim_{V(0) \rightarrow -\infty} \sqrt{I} \cdot \tan \left( t \sqrt{I} + \arctan \left( \frac{V(0)}{\sqrt{I}} \right) \right) = \sqrt{I} \cdot \tan \left( t \sqrt{I} - \frac{\pi}{2} \right) \\
&=  \sqrt{I} \cdot \cot \left( t \sqrt{I} \right) 
\end{align*}


\subsection{Frequency response of the neuron models} \label{app:ThetaModelFrequencyResponse}
The integral is solved like before, but now with the conditions of the spike:
\begin{align*}
T &= \lim_{a \rightarrow \infty} \int_{-a}^{a} \frac{I}{\left(\frac{v}{\sqrt{I}}\right)^2 + 1} \mathop{dv} 
\overset{x = \frac{v}{\sqrt{I}}}{\underset{\mathop{dx} = \frac{dv}{\sqrt{I}}}{=}} 
\lim_{a \rightarrow \infty} \int_{\frac{-a}{\sqrt{I}}}^{\frac{a}{\sqrt{I}}} \frac{I}{x^2 + 1} \mathop{dx}
= \lim_{a \rightarrow \infty} \frac{1}{\sqrt{I}} \arctan(x) \Big \rvert_{\frac{-a}{\sqrt{I}}}^{\frac{a}{\sqrt{I}}} \\
&= \frac{1}{\sqrt{I}} \left( \frac{\pi}{2} - \left( - \frac{\pi}{2} \right) \right)
= \frac{\pi}{\sqrt{I}}
\end{align*}
So the frequency of oscillation is proportional to $\sqrt{I}$. 

\subsection{Jacobian of the Ott-Antonsen manifold}
\subsection{Jacobian of the Ott-Antonsen extended manifold}







