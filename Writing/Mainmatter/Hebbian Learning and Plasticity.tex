% !TEX root = ../main.tex
\newpage
\section{Hebbian Learning and Synaptic Plasticity}
\vspace{1mm}
\begin{quote}
\textsl{When an axon of cell A is near enough to excite a cell B and repeatedly or persistently takes part in firing it, some growth process or metabolic change takes places in one or both cells such that A's efficiency, as one of the cells firing B, is increased.}\cite{Hebb1949}
\end{quote}

This quote from Hebb has influenced the neuroscientific community since 1949. In its essence, Hebb postulated that neurons that \textsl{fire together, wire together}.  
The dynamics of the network occur on a slower time-scale than the dynamics on the network, and both types of dynamics influence each other \cite{AdaptiveNetworks2009}. 

\subsection{Hebbian Learning}
Lorem ipsum dolor sit amet, consectetur adipiscing elit. Quisque nisl eros, 
pulvinar facilisis justo mollis, auctor consequat urna. Morbi a bibendum metus. 
Donec scelerisque sollicitudin enim eu venenatis. Duis tincidunt laoreet ex, 
in pretium orci vestibulum eget.


\subsection{Anti-hebbian learning}
Lorem ipsum dolor sit amet, consectetur adipiscing elit. Quisque nisl eros, 
pulvinar facilisis justo mollis, auctor consequat urna. Morbi a bibendum metus. 
Donec scelerisque sollicitudin enim eu venenatis. Duis tincidunt laoreet ex, 
in pretium orci vestibulum eget.


\subsection{Spike-timing dependant plasticity}
The process that allows neurons to adjust the strength of their synapses is called \textsl{synaptic plasticity}. Over time more (or less) neurotransmitters can be released (potentiation/depression). We will model this behavior by allowing the coupling strength to change over time, according to a rule. As we are interested in the long term effects on synchronization of the network, short-term potentiation and depression will not be considered \cite{MathFoundationNeuroscience}.
One specific rule that determines the evolution of coupling strength over time is \textsl{spike-timing-dependent plasticity} (\STDP), where the relative timing of action potentials from the pre- and postsynaptic neuron determine causality. This is a temporal interpretation of Hebbian learning \cite{Kempter1999, Gerstner2002}. 
If the postsynaptic neuron fires right after the presynaptic neuron, then we will strengthen the coupling strength from post- to presynaptic neuron, and vice versa. The magnitude of change is modulated by an asymmetric biphasic learning window around pulses originating from the postsynaptic neuron. Asymmetric because the peak is not situated at 0 and the integral over the window is generally positive, biphasic because this allows both to strengthen and weaken coupling strengths \cite{Gerstner2002}. Recently, triphasic learning windows have been used to account for when it takes too long for the postsynaptic neuron to fire, and thus to decorrelate the relation between neurons. These learning windows are curves that were fitted to experimental data of the cortex and the hippocampus \cite{ChrolCannon2014}.
This approach simplifies modeling the neuronal back-propagation, where another pulse is generated as an echo of the action potential which travels through the neuron dendrites (so, backwards). This behaviors is believed to adjust the presynaptic weights, though it is a controversial subject \cite{Gerstner2002}.

 In recent years, criticism on \STDP has been growing, as experimental data has shown that \STDP is usually accompanied by homeostatic plasticity of the neuron excitability and the synaptic strengths. Processes like \textsl{intrinsic plasticity}, where one neuron's excitability changes over time as to self-regulate sensitivity to incoming action potentials, or \textsl{synaptic scaling}, where synapse characteristics are adjusted in unison to counteract positive feedback loops, have proven to stabilize the firing rate \cite{ChrolCannon2014, Kirkwood2019}. We can model intrinsic plasticity by adjusting the neuron's excitability as the inverse of the firing rate: he more spikes that a neuron will receive, the less affected it is \cite{LiXueSong2017}. An observed phenomenon is that the excitability evolves together with the coupling strength, but that at the extremes this relation reverses \cite{Debanne2017, Debanne2018}.  These types of plasticities should be relatively easy to implement but have no impact on the network topology.


\subsection{Intrinsic plasticity}
Lorem ipsum dolor sit amet, consectetur adipiscing elit. Quisque nisl eros, 
pulvinar facilisis justo mollis, auctor consequat urna. Morbi a bibendum metus. 
Donec scelerisque sollicitudin enim eu venenatis. Duis tincidunt laoreet ex, 
in pretium orci vestibulum eget.
