% !TEX root = ../main.tex
\newpage
\section{Introduction} \label{sec:Introduction}
%\subsection{Motivation for this work}
In 2013, the European Union launched one of the largest funded scientific project ever. With the Human Brain Project \cite{humanbrainproject}, scientists and researchers aimed to reconstruct the human brain through supercomputer-based models and to advance neuroscience, medicine, and computing. Across the globe different fields of science are drawing inspiration from the human brain. \\

One such approach is to model the behaviour of biological neurons and to quantify the information processes in the brain from stimuli from the senses or from electrical and chemical processes in the body. A given neuron receives hundreds of impulses in the form of neurotransmitters from its neighbours, almost exclusively on its dendrites and cell body. These stimuli add up to an excitatory or inhibitory influence on the electrical membrane potential of the neuron, so that the potential \textsl{spikes} when excitation is higher than an internal threshold. This explosion of electrical activity is called the \textsl{action potential} \cite{IntroductionModelingDynamics}. At the neuron's synapse, the action potential is converted into a chemical signal again, in the form of a neurotransmitter release, and the neuron joins the interneuronal communication \cite{ActionPotentialsAndSynapses}. %Most neurons in the central nervous system use either the excitatory neurotransmitter glutamate (AMPA or NMDA) or the inhibitory neurotransmitter GABA \cite{MathFoundationNeuroscience, Zhang2012}. 
We will speak of the presynaptic neuron as the neuron that sends a signal and of the postsynaptic neuron as the neuron that receives a signal. \\

%At this point, the neuron releases its own neurotransmitter and joins the interneuronal communication \cite{IntroductionModelingDynamics}. 


The neuron dynamics are largely captured by this spiking behaviour, on which most efforts of finding a description have been concentrated.
In 1952, Hodgkin and Huxley formulated a mathematical model for the action potentials in neurons, using a set of nonlinear differential equations that approximates the electrical characteristics of the neuron elements. The authors were awarded the 1963 Nobel Prize in Physiology or Medicine \cite{nobel1963} for their work.\\

As the human brain contains more than 100 billion neurons \cite{Herculano2009} it is unfeasible to study complex models at this scale, setting aside the complexity of the Hodgkin-Huxley model. The topology of neuronal networks displays traits of small-worldness, wiring optimisation, and heterogeneous degree distributions \cite{Bullmore2010}, for which it is difficult to pin down one type of network architecture. Through the mean-field reduction (\MFR) \cite{OttAntonsen2008, OttAntonsen2009, OttAntonsen2010}, one can reduce a large network of indistinguishable neurons to a low-dimensional dynamical system, described by the attraction of a mean-field variable to a reduced manifold.
In this work we will study the \MFR of different types of networks of pulse-coupled Theta neurons using the recent advancements on networks with arbitrary topologies \cite{OttAntonsen2017}. \\


%\subsection{Modelling neuronal behaviour}
%Neurons communicate through electrical and chemical signals, in the form of action potentials and neurotransmitters respectively. We will speak of the presynaptic neuron as the neuron that sends a signal and of the postsynaptic neuron as the neuron that receives a signal. %When the membrane voltage of a presynaptic neuron reaches an internal threshold, the neuron \textsl{spikes} (or \textsl{fires}) and an electrical signal travels down the neuron axons \cite{IntroductionModelingDynamics}. 

%As signals are converted from electrical to chemical and back, we can easily model the communication process as entirely electrical, consisting solely of action potentials. %We will use a pulse-shaped curve, modulated by a dynamic variable of the neuron (for example the phase angle for the Theta neuron). The magnitude of the response of the postsynaptic neuron can be modeled by a coupling strength. The inhibitory/excitatory behavior can be modeled by allowing a negative/positive coupling strength between neurons \cite{Luke2013, Martens2020, Montbrio2015, OttAntonsen2017}.
% Another electrical communication process occurs when current flows between neurons proportional to the difference in their membrane voltages, known as \textsl{gap junctions}. We will follow the work that has been done on pulse-shaped coupling.
%We will avoid modeling the two communication modes at the same time, due to tractability \cite{Martens2020}. 

Neurons have the ability to adjust the intensity of their response to in- and outbound signals depending on their activity in a process called \textsl{synaptic plasticity}. We can observe how neurons release a different quantity of neurotransmitters, or relocate the neurotransmitter receptors. The average intensity of the response is referred to as the \textsl{synaptic strength}. Plastic changes to the brain structure are believed to be the foundation of learning and the memory. One can quantify and model these changes to the network topology using Hebbs postulate on the correlation of neuron activity in the network \cite{Hebb1949}. Changes to the network topology are easily modeled as changes to the synaptic strength.\\

The work presented here is thus two-fold:  we study the dynamics of pulse-coupled neurons \textsl{on} networks, and the dynamics \textsl{of} such networks when their topology evolves over time. First, we will treat the existing theory on neuron models, network topologies and the \MFR in detail, before investigating the synchronisation dynamics of different neuronal networks. Using the acquired knowledge, the theory on learning dynamics is treated, upon which more advanced learning models are formulated and compared to each other.

