% !TEX root = ../main.tex
\newpage
\section{Introduction} \label{sec:Introduction}
%\subsection{Motivation for this work}
In 2013, one of the largest scientific projects ever funded by the European Union was launched. With the Human Brain Project \cite{humanbrainproject}, scientists and researchers aimed to reconstruct the human brain through supercomputer-based models and to advance neuroscience, medicine, and computing. Across the globe different fields of science are drawing inspiration from the human brain, through different approaches. \\
One such approach is to model the behaviour of biological neurons and to quantify the information processes in the brain from stimuli from the senses or from electrical and chemical processes in the body. A given neuron receives hundreds of impulses in the form of neurotransmitters, almost exclusively on its dendrites and cell body. These stimuli add up to an excitatory or inhibitory influence on the membrane potential of the neuron, so that the potential spikes when excitation is higher than an internal threshold. At this point, the neuron releases its own neurotransmitter and joins the interneuronal communication \cite{IntroductionModelingDynamics}. The neuron dynamics are largely captured by this spiking behaviour, on which most efforts have been concentrated.
In 1952, Hodgkin and Huxley described a mathematical model for the action potentials in neurons, using a set of nonlinear differential equations that approximates the electrical characteristics of the neuron elements. In 1963 the authors were awarded the Nobel Prize in Physiology or Medicine \cite{nobel1963} for their work.\\

As the human brain contains more than 100 billion neurons \cite{Herculano2009} it is unfeasible to study complex models at this scale. The topology of neuronal networks displays traits of small-worldness, wiring optimisation, and heterogeneous degree distributions \cite{Bullmore2010}, for which it is difficult to pin down one type of network architecture. Through the mean-field reduction (\MFR) proposed in \cite{OttAntonsen2008} one can reduce a large network of indistinguishable neurons to a low-dimensional dynamical system, described by the attraction of a mean-field variable to a reduced manifold.
In this work we will study the \MFR of different types of networks of coupled Theta neurons using the generalisations found in \cite{OttAntonsen2017}. \\


%\subsection{Modelling neuronal behaviour}
Neurons communicate through \textsl{synapses} with electrical and chemical signals, in the form of action potentials and neurotransmitters respectively. We will speak of the presynaptic neuron as the neuron that sends a signal and of the postsynaptic neuron as the neuron that receives a signal. When the membrane voltage of a presynaptic neuron reaches an internal threshold, the neuron \textsl{spikes} (or \textsl{fires}) and an electrical signal travels down the neuron axons \cite{IntroductionModelingDynamics}. At the synapse, the electrical signal is converted into a chemical signal in the form of a neurotransmitter release of the presynaptic neuron, upon which the postsynaptic neuron receives the neurotransmitters and constructs its own electrical signal \cite{ActionPotentialsAndSynapses}. Most neurons in the central nervous system use either the excitatory neurotransmitter glutamate (AMPA or NMDA) or the inhibitory neurotransmitter GABA \cite{MathFoundationNeuroscience, Zhang2012}. \\

%As signals are converted from electrical to chemical and back, we can easily model the communication process as entirely electrical, consisting solely of action potentials. %We will use a pulse-shaped curve, modulated by a dynamic variable of the neuron (for example the phase angle for the Theta neuron). The magnitude of the response of the postsynaptic neuron can be modeled by a coupling strength. The inhibitory/excitatory behavior can be modeled by allowing a negative/positive coupling strength between neurons \cite{Luke2013, Martens2020, Montbrio2015, OttAntonsen2017}.
% Another electrical communication process occurs when current flows between neurons proportional to the difference in their membrane voltages, known as \textsl{gap junctions}. We will follow the work that has been done on pulse-shaped coupling.
%We will avoid modeling the two communication modes at the same time, due to tractability \cite{Martens2020}. 

The process that allows neurons to adjust the strength of their synapses is called \textsl{synaptic plasticity}. This makes neurons more susceptible to each others behaviour, and results in increased (or inhibited) synchronisation of brain waves. Using Hebbs postulate \cite{Hebb1949}, on the behaviour of the neuronal network to form new or strengthen connections through the synaptic strength, one can quantify and model those changes to the network topology. \\

The work presented here is thus two-fold:  we study the dynamics of pulse-coupled networks \textsl{on} networks, and the dynamics \textsl{of} such networks when they evolve over time. 