% !TEX root = ../main.tex
\newpage
\section{Introduction}
\noindent In 2013, one of the largest scientific projects ever funded by the European Union was launched. With the Human Brain Project \cite{humanbrainproject}, scientists and researchers aimed to reconstruct the human brain through supercomputer-based models and to advance neuroscience, medicine, and computing. Across the globe different fields of science are drawing inspiration from the human brain, through different approaches. \\
One such approach is to model the behaviour of biological neurons and to quantify the information processes in the brain from stimuli from the senses or from electrical and chemical processes in the body. A given neuron receives hundreds of impulses in the form of neurotransmitters, almost exclusively on its dendrites and cell body. These stimuli add up to an excitatory or inhibitory influence on the membrane potential of the neuron, so that the potential spikes when excitation is higher than an internal threshold. At this point, the neuron releases its own neurotransmitter and joins the interneuronal communication \cite{IntroductionModelingDynamics}. The neuron dynamics are largely captured by this spiking behaviour, on which most efforts have been concentrated.
In 1952, Hodgkin and Huxley described a mathematical model for the action potentials in neurons, using a set of nonlinear differential equations that approximates the electrical characteristics of the neuron elements. In 1963 the authors were awarded the Nobel Prize in Physiology or Medicine \cite{nobel1963} for their work.\\
As the human brain contains more than 100 billion neurons \cite{Herculano2009} it is unfeasible to study complex models at this scale. The topology of neuronal networks displays traits of small-worldness, wiring optimisation, and heterogeneous degree distributions \cite{Bullmore2010}, for which it is difficult to pin down one type of network architecture. Through the mean-field reduction (\MFR) proposed in \cite{OttAntonsen2008} one can reduce a large network of indistinguishable neurons to a low-dimensional dynamical system, described by the attraction of a mean-field variable to a reduced manifold.
In this paper we will study the \MFR of different types of networks of coupled theta neurons using the generalisations found in \cite{OttAntonsen2017}.

We distinguish between two types of dynamics: we study the dynamics of pulse-coupled networks \textsl{on} networks, and the dynamics \textsl{of} such networks is how they evolve over time. The dynamics of the network occur on a different (slower) time-scale than the dynamics on the network. Both types of dynamics influence each other \cite{AdaptiveNetworks2009}.
