%% ------ Packages ------ %%
% Related to the document setup:
\documentclass[12pt, a4paper]{extarticle}
\usepackage[a4paper, top = 2.4cm, bottom = 2.4 cm, right= 2.1cm, left= 2.1cm]{geometry}
\usepackage[english]{babel}
\renewcommand\familydefault{\sfdefault}
\usepackage{amsmath}
\usepackage{mathtools}
\usepackage{amsfonts}
\usepackage{amssymb}

\usepackage[T1]{fontenc}
\setlength\parindent{0pt}
\usepackage{multicol}
\usepackage{xspace}


% Colouring the references
\usepackage{hyperref}
\usepackage{cleveref}
\usepackage[dvipsnames]{xcolor}
\pagecolor{white}
\newcommand\myshade{85}
\colorlet{mylinkcolor}{violet}
\colorlet{mycitecolor}{YellowOrange}
\definecolor{myurlcolor}{rgb}{ 0, 0.4470, 0.6410}

\hypersetup{
  linkcolor  = mylinkcolor!\myshade!black,
  citecolor  = mycitecolor!\myshade!black,
  urlcolor   = myurlcolor!\myshade!black,
  colorlinks = true,
}

% Nomenclature
%\usepackage{nomencl}
%\makenomenclature
%%\renewcommand{\nomname}{List of symbols}
%\renewcommand{\nompreamble}{\noindent Definitions of the nomenclature}
%\newlength{\nomitemorigsep}
%\setlength{\nomitemorigsep}{\nomitemsep}
%\setlength{\nomitemsep}{-\itemsep}

% Section properties redefinitions
\makeatletter
\renewcommand\section{\@startsection {section}{1}{\z@}{3ex }{0.1ex } {\normalfont\large\bfseries}}
\renewcommand\subsection{\@startsection{subsection}{1}{\z@}{2ex }{0.1ex } {\normalfont \bfseries}}
\renewcommand\subsubsection{\@startsection{subsubsection}{3}{\z@}{-1.5ex\@plus -1ex \@minus -.2ex}{.5ex}{\normalfont}}
\makeatother

% Graphics interfaces:
\usepackage{graphicx}
\usepackage{tikz}
\tikzset{every picture/.style={line width=1pt}}
\usepackage{float}
\usepackage{subcaption} 
\usepackage[font=small,aboveskip=3pt, belowskip=-0pt]{caption}


% Headers and footers:
\usepackage{url}
\usepackage{footnote}
\usepackage{fancyhdr}
\pagestyle{fancy} 
\fancyhf{} 
\renewcommand{\headrulewidth}{0pt}
\newcommand{\mainmatter}{\clearpage \cfoot{\thepage\ of \pageref{LastPage}}
\setcounter{page}{1}
\pagenumbering{arabic}}
\usepackage{comment}

% Sort the bibliography
\usepackage[sort]{cite}
\setlength\columnsep{15pt}

% Command definitions:
\newcommand{\dtheta}{{\dot{\theta}}\xspace}
\newcommand{\ic}{{\rm{i}}\xspace}

\renewcommand{\Re}{{\operatorname{Re}}\xspace}
\newcommand{\R}{{\rm I\!R}\xspace}
\newcommand{\C}{{\mathbb{C}}\xspace}
\newcommand{\T}{{\mathbb{T}}\xspace}

\renewcommand{\k}{{\boldsymbol{k}}\xspace}
\newcommand{\kacc}{{\boldsymbol{k'}}\xspace}
\newcommand{\kmean}{{\langle k\rangle}\xspace}
\newcommand{\kin}{{k^{\rm in}}\xspace}
\newcommand{\kout}{{k^{\rm out}}\xspace}
\newcommand{\degree}{{\rm deg}\xspace}

\newcommand{\tp}{{t^{\prime}}\xspace}
\newcommand{\tpp}{{t^{\prime \prime}}\xspace}

\newcommand{\Sin}{{S^{\rm in}}\xspace}
\newcommand{\Sout}{{S^{\rm out}}\xspace}

\newcommand{\mfr}{{\textsl{MFR}}\xspace}
\newcommand{\SNIC}{{\textsl{SNIC}}\xspace}
\newcommand{\PSR}{{\textsl{PSR}}\xspace}
\newcommand{\PSS}{{\textsl{PSS}}\xspace}
\newcommand{\CPW}{{\textsl{CPW}}\xspace}
\newcommand{\STDP}{{\textsl{STDP}}\xspace}
\newcommand{\LTP}{{LTP}\xspace}
\newcommand{\LTD}{{LTD}\xspace}

\def\matlab{\textsc{Matlab}\xspace}


%% ------ Front page ------ %%

\begin{document}

\mainmatter

Synaptic Plasticity \today \\
Msc Thesis - Dynamics of adaptive neuronal networks. \\
Simon Aertssen (s181603), \today \\ 

\section{Spike Timing Dependant Plasticity}
Write about STDP, see the 'Postulates.pdf'. \\

With spike timing dependant plasticity we want to study how two neurons change their synaptic strength based on the time delay between spikes. When $\theta(t)_j > \pi$ we say that the neuron $\theta_j$ spikes at time $t$. 
Let us say that $\theta_i$ spikes at time $t_i$ and $\theta_j$ spikes at $t_j$. Taking the time difference $\Delta t_{ij}$ as $t_j - t_i$, we can say that when $\Delta t_{ij} > 0$ the spikes are correlated (there exists a temporally causal relation), and we can model an increase in synaptic strength of the connection $K_{ij}$ from $\theta_i$ to $\theta_j$. In the same fashion we can decrease $K_{ji}$ when $\Delta t_{ij} < 0$ as there is no causal relation. \\
The functions $W(t)$ that relate $\Delta t_{ij}$ to $\Delta K_{ij}$ are called \textsl{learning windows},  as they define a range in which $K_{ij}$ is able to adapt. When signals between neurons show a very large time difference (negative or positive) we do not expect them to be correlated. Because the learning windows are generally not symmetrical we can also expect an asymmetrical adjacency matrix.\\
Another characteristic is the integral over the learning window. A window with a negative integral directs synaptic strengths mostly towards inhibitory behaviour, and vice versa with a positive integral. An integral of zero would mean that both inhibitory and excitatory synapses are stimulated equally.\\


\subsection{Method}
Now that we have a feeling of how \STDP works, we need to formulate the behaviour exactly. Following the notation in \cite{Kempter1999}, we will denote the spike train coming from each neuron $\theta_i$ as $S_i^{\rm out}(t) = \sum_{n} \delta (t-t_{i}^{n})$, where $t_{i}^{n}$ is the time that $\theta_i$ has fired. Similarly, we will denote the spike train coming into each neuron $\theta_i$ as $S_i^{\rm in}(t) = \sum_{f} \delta (t-t_{i}^{f})$. Now we can say that the synaptic strengths are adjusted as:
\begin{align}
\Delta K_{ij} &= \int_{t}^{t+\mathcal{T}} w^{\rm{out}} S_i^{\rm out}(\tau) + w^{\rm{in}} S_{j}^{\rm {in}}(\tau) \mathrm{d}\tau
+ \iint_{t}^{t+\mathcal{T}} W( \tau^\prime - \tau) S_{i}^{\rm out}(\tau) S_{j}^{\rm in}( \tau^\prime) \mathrm{d} \tau \mathrm{d} \tau^\prime
\label{eq:KempterFormulation1} \\
&= \sum_{t_i^{n}\in \mathcal{T}} w^{\mathrm{out}} + \sum_{t_{j}^{f} \in \mathcal{T}} w^{\mathrm{in}} + \sum_{t_{j}^{f}, t_i^{n} \in \mathcal{T}} W\left(t_{j}^{f}-t_i^{n}\right) \label{eq:KempterFormulation2}
\end{align}
\eqref{eq:KempterFormulation1} is the most general formulation of \STDP and includes a small change in weights when a signal is merely received. Whether this is a process that influences \STDP is still under investigation, but we will neglect the contribution.


\subsection{Biphasic Learning windows}
\subsubsection{Kempter 1999}
\begin{align}
W(t)_K = \eta
\begin{cases}
\left[A_{p}\left(1-\frac{t}{\tilde{\tau}_{p}}\right)+A_{n}\left(1-\frac{t}{\tilde{\tau}_{n}}\right)\right] \cdot \exp \left( \frac{t}{\tau_{\rm syn}} \right) & \text{for } t \leq 0 \\
A_{p} \cdot \exp \left(-\frac{t}{\tau_{p}}\right) + A_{n} \cdot \exp \left(-\frac{t}{\tau_{n}} \right) & \text{for } t > 0
\end{cases} \label{eq:learningwindowKempter1999}
\end{align}
Here $t$ is the delay between presynaptic spike arrival and postsynaptic firing, $\eta$ is a small learning parameter and all $\tau$ are time constants. Numerical values are usually  $\eta = 0.05$, $\tau_{\rm syn} = 5$ ms, $\tau_{p} = 1$ ms, $\tau_{n} = 20$ ms and $A_p = 1$ and $A_{n} = -1$. $\tilde{\tau}_{p} \equiv \tau_{\rm syn} \tau_{p} / (\tau_{\rm syn} + \tau_{p})$ and $\tilde{\tau}_{n} \equiv \tau_{\rm syn} \tau_{n} / (\tau_{\rm syn} + \tau_{n})$. \\
$\int W(s)_K \mathrm{d}s = 2.56 \times 10^{-4}$.


\subsubsection{Song 2000}
The first formulation of \STDP as a mathematical model was in \cite{Song2000}. 
\begin{align}
W(t)_S =
\begin{cases}
A_{p} \cdot \exp \left(\frac{-t}{\tau_p}\right) & \text{for } s > 0 \\
A_{n} \cdot \exp \left(\frac{t}{\tau_n}\right)  & \text{for } s \leq 0
\end{cases} \label{eq:learningwindowKempter1999}
\end{align}
where we will use the values as proposed in \cite{ChrolCannon2012}: $A_p = 0.1$, $A_n = -0.12$ and $\tau_p = \tau_n = 20$. $\int W(s)_S \mathrm{d}s = -3.70 \times 10^{-4}$.

\begin{figure}[H]
\centering
\includegraphics[width = \textwidth]{../Figures/LearningWindowsBiphasic.pdf}
\caption{Two different biphasic learning windows. We can see how in $W(t)_s$ a larger weight is put on the anti-Hebbian learning.}
\label{fig:LearningWindowsBiphasic}
\end{figure}


\subsection{Triphasic Learning windows}

\subsubsection{Chrol-Cannon 2012}
\begin{align}
W(t)_C = A_{p} \cdot \exp \left(\frac{-\left(t - 15 \right)^{2}}{ \tau_{p}}\right) - A_{n} \cdot \exp \left(\frac{-\left(t - 20\right)^{2}}{ \tau_{n}}\right)  \label{eq:learningwindowChrolCannon2012}
\end{align}
where $A_{p}=0.23$, $A_{n}=0.15$, $\tau_{p}=200$ and $\tau_n = 2000$. $\int W(s)_C \mathrm{d}s = -60.0 \times 10^{-4}$.

\subsubsection{Waddington 2014}
\begin{align}
W(t)_W =  A \left[1-\frac{\left(t-\alpha\right)^{2}}{\alpha^{2}}\right] \cdot \exp \left(\frac{-\left|t - \alpha\right|}{\alpha}\right) \label{eq:learningwindowWaddington2014}
\end{align}
We will use $A = 0.1$ and $\alpha = 4.0$ ms . $\int W(s)_W \mathrm{d}s = -8.0 \times 10^{-4}$.

\begin{figure}[H]
\centering
\includegraphics[width = \textwidth]{../Figures/LearningWindowsTriphasic.pdf}
\caption{Two different triphasic learning windows. We can see how in $W(t)_C$  a much higher penalty is given to signals that arrive too early or too late. The moments at which signals should peak also differ by about 10 ms.}
\label{fig:LearningWindowsTriphasic.pdf}
\end{figure}


\subsection{Custom learning windows}
Some of the behaviour we want to observe is currently not accounted for: synaptic strengths are unbounded and are generally nonzero. How can we model the behaviour where synaptic strengths can also settle on being zero? Perhaps a better idea than updating the synaptic strength by adding a new value, we can scale it.
\begin{itemize}
\item When $\Delta t_{ij}$ is very large (both positive and negative) we expect no change in the synaptic strength: $W(-\infty) = W(-\infty) = 0$. 
\item We expect a specific positive time delay to yield the most amount of synaptic strengthening: $\Delta t_{\rm best} =\underset{x}{\arg \max} W(t)$
\end{itemize}



\bibliographystyle{utphys}
\small{\bibliography{references}}

\label{LastPage}~

\end{document}
