%% ------ Packages ------ %%
% Related to the document setup:
\documentclass[12pt, a4paper]{extarticle}
\usepackage[a4paper, top = 2.4cm, bottom = 2.4 cm, right= 2.1cm, left= 2.1cm]{geometry}
\usepackage[english]{babel}
\renewcommand\familydefault{\sfdefault}
\usepackage{amsmath}
\usepackage{mathtools}
\usepackage{amsfonts}
\usepackage{amssymb}

\usepackage[T1]{fontenc}
\setlength\parindent{0pt}
\usepackage{multicol}
\usepackage{xspace}


% Colouring the references
\usepackage{hyperref}
\usepackage{cleveref}
\usepackage[dvipsnames]{xcolor}
\pagecolor{white}
\newcommand\myshade{85}
\colorlet{mylinkcolor}{violet}
\colorlet{mycitecolor}{YellowOrange}
\definecolor{myurlcolor}{rgb}{ 0, 0.4470, 0.6410}

\hypersetup{
  linkcolor  = mylinkcolor!\myshade!black,
  citecolor  = mycitecolor!\myshade!black,
  urlcolor   = myurlcolor!\myshade!black,
  colorlinks = true,
}

% Nomenclature
%\usepackage{nomencl}
%\makenomenclature
%%\renewcommand{\nomname}{List of symbols}
%\renewcommand{\nompreamble}{\noindent Definitions of the nomenclature}
%\newlength{\nomitemorigsep}
%\setlength{\nomitemorigsep}{\nomitemsep}
%\setlength{\nomitemsep}{-\itemsep}

% Section properties redefinitions
\makeatletter
\renewcommand\section{\@startsection {section}{1}{\z@}{3ex }{0.1ex } {\normalfont\large\bfseries}}
\renewcommand\subsection{\@startsection{subsection}{1}{\z@}{2ex }{0.1ex } {\normalfont \bfseries}}
\renewcommand\subsubsection{\@startsection{subsubsection}{3}{\z@}{-1.5ex\@plus -1ex \@minus -.2ex}{.5ex}{\normalfont}}
\makeatother

% Graphics interfaces:
\usepackage{graphicx}
\usepackage{tikz}
\tikzset{every picture/.style={line width=1pt}}
\usepackage{float}
\usepackage{subcaption} 
\usepackage[font=small,aboveskip=3pt, belowskip=-0pt]{caption}


% Headers and footers:
\usepackage{url}
\usepackage{footnote}
\usepackage{fancyhdr}
\pagestyle{fancy} 
\fancyhf{} 
\renewcommand{\headrulewidth}{0pt}
\newcommand{\mainmatter}{\clearpage \cfoot{\thepage\ of \pageref{LastPage}}
\setcounter{page}{1}
\pagenumbering{arabic}}
\usepackage{comment}

% Sort the bibliography
\usepackage[sort]{cite}
\setlength\columnsep{15pt}

% Command definitions:
\newcommand{\dtheta}{{\dot{\theta}}\xspace}
\newcommand{\ic}{{\rm{i}}\xspace}

\renewcommand{\Re}{{\operatorname{Re}}\xspace}
\newcommand{\R}{{\rm I\!R}\xspace}
\newcommand{\C}{{\mathbb{C}}\xspace}
\newcommand{\T}{{\mathbb{T}}\xspace}

\renewcommand{\k}{{\boldsymbol{k}}\xspace}
\newcommand{\kacc}{{\boldsymbol{k'}}\xspace}
\newcommand{\kmean}{{\langle k\rangle}\xspace}
\newcommand{\kin}{{k^{\rm in}}\xspace}
\newcommand{\kout}{{k^{\rm out}}\xspace}
\newcommand{\degree}{{\rm deg}\xspace}

\newcommand{\tp}{{t^{\prime}}\xspace}
\newcommand{\tpp}{{t^{\prime \prime}}\xspace}

\newcommand{\Sin}{{S^{\rm in}}\xspace}
\newcommand{\Sout}{{S^{\rm out}}\xspace}

\newcommand{\mfr}{{\textsl{MFR}}\xspace}
\newcommand{\SNIC}{{\textsl{SNIC}}\xspace}
\newcommand{\PSR}{{\textsl{PSR}}\xspace}
\newcommand{\PSS}{{\textsl{PSS}}\xspace}
\newcommand{\CPW}{{\textsl{CPW}}\xspace}
\newcommand{\STDP}{{\textsl{STDP}}\xspace}
\newcommand{\LTP}{{LTP}\xspace}
\newcommand{\LTD}{{LTD}\xspace}

\def\matlab{\textsc{Matlab}\xspace}


%% ------ Front page ------ %%

\begin{document}

\mainmatter

Status meeting \today \\
Msc Thesis - Dynamics of adaptive neuronal networks. \\
Simon Aertssen (s181603), \today \\ 

\section{Writing out the whole system}
The network dynamics are described as follows:
\begin{align}
\frac{d \theta_i}{d t} &= (1 - \cos\theta_i) + (1 + \cos\theta_i)\cdot\left(\eta_i + I_i \right) \qquad \theta_i \in \R^N \nonumber \\
I_{i} &= \frac{\kappa}{\langle k\rangle} \sum_{j=1}^{N} A_{i j} P_{n}\left(\theta_{j}\right) \label{eq:FullThetaNeuronNetwork} \\
P_{n}\left(\theta_{j}\right) &= (1 - \cos\theta_j) \nonumber
\end{align}

We observe synchronization through the order parameter
\begin{align}
Z(t) = \frac{1}{N} \sum_{j=1}^N e^{\ic\theta_j}  \qquad Z(t) \in \C \label{eq:orderparameter}
\end{align}

For a fixed degree network it has been proven that the order parameter follows:
\begin{align}
\dot{Z}(t)= -\ic \frac{(Z-1)^2}{2}+\frac{(Z+1)^2}{2} \cdot \left(-\Delta+ \ic\eta_{0}
+ \ic \kappa \cdot \left(1+\frac{Z^{2} + \overline{Z}^{2} }{6} - \frac{4}{3} \Re(Z)\right)\right) \label{eq:MeanField}
\end{align}
    
For an arbitrary network the order parameter follows a trajectory per degree. When we assemble the whole expression for the Ott-Antonsen manifold as found in \cite{OttAntonsen2017} with $H_2(\k,t)$ as in \cite{Martens2020}, we obtain the following:
\begin{align}
\frac{\partial z(\k, t)}{\partial t} &= -\ic \frac{(z(\k, t)-1)^{2}}{2} + \frac{(z(\k, t)+1)^{2}}{2} \cdot I(\k) \qquad z(\k,t) \in \C^{M_\k} \nonumber \\
I(\k) &= -\Delta(\k) + \ic \eta_{0}(\k) + i d_{n} \kappa \cdot H_n(\k,t) \label{eq:OttAntonsenSystemFull} \\
%H_n(\k,t) &= \frac{a_n}{\kmean} \sum_{\kacc} P\left(\kacc\right) a\left(\kacc \rightarrow \k\right) \cdot \left[A_{0}+\sum_{p=1}^{n} A_{p}\left(z\left(\kacc, t\right)^{p} + \overline{z}\left(\kacc, t\right)^{p}\right)\right] 
H_2(\k,t) &= \frac{1}{\kmean} \sum_{\kacc} P\left(\kacc\right) a\left(\kacc \rightarrow \k\right) \cdot \left( 1 + \frac{z(\kacc, t)^2 + \overline{z}(\kacc, t)^2}{6} - \frac{4}{3} \Re(z(\kacc, t)) \right) \nonumber
\end{align}

We can then find the mean field dynamics through
\begin{align}
\overline{Z}(t) &= \frac{1}{N} \sum_{\k} P(\k) z(\k, t) \qquad \overline{Z}(t) \in \C \label{eq:OttAntonsenMeanField}
\end{align}

It is important to notice that in \eqref{eq:OttAntonsenSystemFull} and \eqref{eq:OttAntonsenMeanField} we actually compute an inner vector product, which is non-commutative for complex numbers:
\begin{align}
a \cdot b = \overline{b \cdot a} \qquad a, b \in \C^r
\end{align}
This is the result of the \textsl{Conjugate} or \textsl{Hermitian} symmetry of the inner product.


\section{Initial conditions}
As the systems in \cref{eq:FullThetaNeuronNetwork,eq:orderparameter,eq:MeanField,eq:OttAntonsenSystemFull,eq:OttAntonsenMeanField} describe the same dynamics for fully connected networks, it is important to be able to transform initial conditions between systems. If we have the same initial conditions, then all systems will predict the same behaviour. We will only map everything to $\C$.\\
The following maps can be used to transform the initial conditions, but as they do not give any qualitative information on the dynamics or distributions of the variables, they are not valid for transforming between dynamics. We discard $t=0$ for clarity.
\begin{alignat*}{3}
&f(Z \rightarrow \theta_i) \quad &&\C \rightarrow \R^N \qquad &&\theta_i = -\ic \cdot \log \left( Z \right) \: \forall i\\
&f(z(\k) \rightarrow Z) \quad &&\C^{M_\k} \rightarrow \C \qquad &&Z = \frac{1}{N} \sum_{\k} P(\k) z(\k, t) \\
&f(\theta_i \rightarrow z(\k)) \quad &&\R^N \rightarrow \C^{M_\k} \qquad &&z(\k) = \sum_k e^{\ic \vartheta_{k}} \quad \vartheta_{k} \in {\theta_i: \sum_j^N A_{ij} = \k}\\
&f(\theta_i \rightarrow Z) \quad &&\R^N \rightarrow \C \qquad &&Z = \frac{1}{N} \sum_{j=1}^N e^{\ic\theta_j} \\
&f(Z \rightarrow z(\k)) \quad &&\C^{M_\k} \rightarrow \C \qquad &&z(\k) = \overline{\frac{Z \cdot n(\k)}{P(\k)}} \\
&f(z(\k) \rightarrow \theta_i) \quad &&\C^{M_\k} \rightarrow \R^N \qquad &&\theta_i = \overline{\frac{Z \cdot n(\k)}{P(\k)}}
\end{alignat*}

\begin{align}
\overline{Z}(t) &= \frac{1}{N} \sum_{k} P(k) z(k, t) = \frac{1}{N} \sum_{k \in \boldsymbol{k}} P(k) \sum_{l \in \left\{\sum_j A_{ij} = k\right\}} e^{\mathrm{i}\theta(t)_l}
\end{align}
and then just take $\overline{Z}(0)$ from
\begin{align}
\overline{Z}(t) &= \frac{1}{N} \sum_{k \in \boldsymbol{k}} P(k) \frac{1}{P(k)}\sum_{l \in \left\{\sum_j A_{ij} = k\right\}} e^{\mathrm{i}\theta(t)_l}
\end{align}
    

\section{Fixpoint iteration}
In \cite{OttAntonsen2017} a fixpoint iteration is suggested to find attractive fixpoints of the system \eqref{eq:OttAntonsenSystemFull}. If we set $\frac{\partial z(\k, t)}{\partial t} = 0$ we can solve the following system:
\begin{align}
\ic \frac{(z(\k, t)-1)^{2}}{2} &= \frac{(z(\k, t)+1)^{2}}{2} \cdot I(\k) \nonumber \\
\ic \left(\frac{z(\k, t)-1}{z(\k, t)+1}\right)^2 &= I(\k) \nonumber \\
\frac{z(\k, t)-1}{z(\k, t)+1} &\equiv b(\k,t) \nonumber \\
z(\k, t) - 1 &= b(\k,t) z(\k, t) + b(\k,t)  \nonumber \\
z(\k, t) \cdot (1 - b(\k,t)) &= b(\k,t)  + 1\nonumber
\end{align}

We can then obtain the stable equilibria from:
\begin{align}
\ic b(\k,t)^2 = I(\k) \hspace{10mm} z(\k, t)_{\pm} = \frac{1 \pm b(\k,t)}{1 \mp b(\k,t)} \label{eq:fixedpointiterations} 
\end{align}
where the signs are chosen so that $\vert z(\k, t) \vert \leq 1$.


\section{A Newton-Raphson iteration for all fixpoints}
\subsection{Theory behind the method}
The fixpoint iteration only gives us the stable equilibria of the system \eqref{eq:OttAntonsenSystemFull}. We can obtain all equilibria and the Jacobian from a Newton-Raphson iteration. We define the equilibria $\boldsymbol{x^\ast} \in \R^n$ of a multivariate function $\boldsymbol{f}(\boldsymbol{x}) : \R^n \rightarrow \R^n$ with $\boldsymbol{f}(\boldsymbol{x}) = \boldsymbol{0}$. Expanding $\boldsymbol{f}$ as a Taylor series, we obtain:
\begin{align}
f_i(\boldsymbol{x} + \delta \boldsymbol{x}) =f_{i}(\boldsymbol{x}) + \sum_{j=1}^{n} \frac{\partial f_{i}(\boldsymbol{x})}{\partial x_{j}} \delta x_{j}+O\left(\delta \boldsymbol{x}^{2}\right) \approx f_{i}(\boldsymbol{x})+\sum_{j=1}^{n} \frac{\partial f_{i}(\boldsymbol{x})}{\partial x_{j}} \delta x_{j}, \qquad(i=1, \cdots, n)
\end{align}

We can also write this in vector notation, by setting $\boldsymbol{J}(\boldsymbol{x}) = \nabla \boldsymbol{f}(\boldsymbol{x}) = \frac{d}{d\boldsymbol{x}} \boldsymbol{f}(\boldsymbol{x}) \in \R^{n \times n}$ 
\begin{align}
\boldsymbol{f}(\boldsymbol{x}+\delta \boldsymbol{x}) &\approx\left[\begin{array}{c}f_{1}(\boldsymbol{x}) \\ \vdots \\ f_{N}(\boldsymbol{x})\end{array}\right] 
+ \left[\begin{array}{ccc}\frac{\partial f_{1}}{\partial x_{1}} & \cdots & \frac{\partial f_{1}}{\partial x_{N}} \\ \vdots & \ddots & \vdots \\ \frac{\partial f_{N}}{\partial x_{1}} & \cdots & \frac{\partial f_{N}}{\partial x_{N}}\end{array}\right]
\left[\begin{array}{c}\delta x_{1} \\ \vdots \\ \delta x_{N}\end{array}\right] 
=\boldsymbol{f}(\boldsymbol{x})+\boldsymbol{J}(\boldsymbol{x}) \delta \boldsymbol{x} 
\end{align}

By assuming $\boldsymbol{f}(\boldsymbol{x}+\delta \boldsymbol{x}) = 0$ we can find that $\delta \boldsymbol{x} = -\boldsymbol{J}^{-1}( \boldsymbol{x}) \boldsymbol{f}(\boldsymbol{x})$ so that $\boldsymbol{x} + \delta \boldsymbol{x} =  \boldsymbol{x} - \boldsymbol{J}^{-1} (\boldsymbol{x}) \boldsymbol{f}(\boldsymbol{x})$. This expression converges to $\boldsymbol{x^\ast}$. When the equations are nonlinear, the equations converge to the real root as $\boldsymbol{x}_k =  \boldsymbol{x}_k - \boldsymbol{J}^{-1} ( \boldsymbol{x}_k)\boldsymbol{f}(\boldsymbol{x}_k)$. \\

For \eqref{eq:OttAntonsenSystemFull}, we can compute the Jacobian for the diagonal and off-diagonal elements separately. But as $z(\k,t)$ is a complex function, first we need to understand what the derivative of a complex function is. 


\subsection{Derivatives of complex functions}
For $z = x + \ic y \in \C$ and $x,y \in R$ the conjugate is defined as $\overline{z} = x - \ic y$. That means that we can write the real and imaginary parts as:
\begin{align*}
x = \frac{z + \overline{z}}{2} \text{  and   }   y = -\ic \frac{z - \overline{z}}{2}
\end{align*}
Using the chain rule, we can write the partial derivative with respect to $z$ in function of $x$ and $y$ as $x$ and $y$ are functionally independent and find the first Wirtinger operator:
\begin{align*}
\frac{\partial}{\partial z} =\frac{\partial x}{\partial z} \frac{\partial}{\partial x}+\frac{\partial \bar{y}}{\partial z} \frac{\partial}{\partial \overline{y}} 
\longrightarrow \frac{\partial x}{\partial z} = \frac{1}{2} \text{   and   } \frac{\partial y}{\partial z} =  - \frac{\ic}{2} 
\longrightarrow \frac{\partial}{\partial z} = \frac{1}{2}\left(\frac{\partial}{\partial x} - \ic \frac{\partial}{\partial y}\right)
\end{align*}
We note the following properties:
\begin{align*}
\frac{\partial}{\partial z} z = 1 \hspace{10mm} \frac{\partial}{\partial z}\overline{z} = \frac{1}{2}\left(1 - \ic^2 \right) = 0
\end{align*}
Interesting for is the result of the following: 
\begin{align*}
\overline{z}^2 &= (x - \ic y)^2 = x^2 - y^2 - \ic 2xy \\
\frac{\partial}{\partial z} \overline{z}^2 &=  \frac{1}{2}\cdot(2x -\ic 2y - \ic\cdot(-2y -\ic 2x)) = x -\ic y + \ic y -\ic x = 0
\end{align*}



\subsection{Derivatives of the complex mean field equations}
We can now compute derivatives of the complex functions $z(\k,t)$. When setting $z(\k, t) = z_k$ the diagonal elements are found as
\begin{equation}
\begin{aligned}[b]
\frac{\partial}{\partial z_k}\left(\frac{\partial z_k}{\partial t} \right) &= - \ic(z_k - 1) + (z_k + 1) \cdot I(z_k) +  \frac{(z_k+1)^{2}}{2} \cdot \frac{\partial I(z_k)}{\partial z_k} \\
\frac{\partial I(z_k)}{\partial z_k} &= \ic \kappa \cdot \frac{\partial H_{2,k}}{\partial z_k} \\
\frac{\partial H_{2,k}}{\partial z_k} &= \frac{1}{\kmean} P_k a_{kk} \cdot \left(\frac{2 z_k}{6} - \frac{4}{3} \cdot \frac{1}{2} \right) = \frac{1}{\kmean} P_k a_{kk} \cdot\frac{z_{k} - 2}{3}
\end{aligned}
\label{eq:OttAntonsenSystemJacobianDiagonal}
\end{equation}

When setting $z(\kacc, t) = z_{k'}$ the off-diagonal elements are found as
\begin{equation}
\begin{aligned}[b]
\frac{\partial}{\partial z_{k'}}\left(\frac{\partial z_k}{\partial t} \right) &= \frac{(z_k + 1)^{2}}{2} \cdot \frac{\partial I(z_k)}{\partial z_{k'}} \\
\frac{\partial I(z_k)}{\partial z_{k'}} &= \ic \kappa \cdot \frac{\partial H_{2,k}}{\partial z_{k'}} \\
\frac{\partial H_{2,k}}{\partial z_{k'}} &= \frac{1}{\kmean} P_{k'} a_{k'k} \cdot \frac{z_{k'} - 2}{3}
\end{aligned}
\label{eq:OttAntonsenSystemJacobianDiagonal}
\end{equation}


\subsection{Results}
I cannot seem to converge close enough.
\begin{figure}[H]
\centering
\includegraphics[width = \textwidth]{../Figures/ProblemsWithNewtonRaphson.png}
\end{figure}



\bibliographystyle{utphys}
\small{\bibliography{references}}

\label{LastPage}~

\end{document}
